\section{Basic typography}

\subsection{Getting started with {\LaTeX}}
\begin{frame}
	\begin{block}{Document class}
		\lcs{documentclass} specifies the type of the document and basic
		layout options. Available document classes are:
		\begin{description}
			\item[article] for journal articles and small texts
			\item[report] longer text e.g. thesis, small books
			\item[book] with parts and chapters
			\item[letter] for old-school communication
			\item[beamer] for presentations (see later)
		\end{description}
	\end{block}

	\begin{block}{German Layouts}
		For a more feasible german layout the KOMA-Script classes can be
		used\footnote{\url{http://www.ctan.org/pkg/koma-script}}:
		\begin{center}\begin{tabular}{l@{$\rightarrow$}r}
		article & scrartcl\\
		 report & scrreprt\\
		book & scrbook
		\end{tabular}\end{center}
	\end{block}
\end{frame}
\begin{frame}
	\begin{block}{Commands}
		Can be used to modify the text, include images and everything else.
		A command
		\code{{\textbackslash}command[optional argument(s)]\{required
		argument\}}
		always start with a backslash ''\textbackslash'' followed by the
		command name. Optional and/or required arguments may be added.
	\end{block}
	An already known command is the \lcs{documentclass} but now with more
	details:
	\code{\textbackslash{}documentclass[9pt,a4paper,twosided,draft]\{article\}}
\end{frame}
\begin{frame}
	\begin{block}{{\LaTeX} groups}
		Some (low-level) arguments do not take arguments but effect more than
		you would expect
		\codex{Without grouping}{commands-without-restriction}
		This can be changed by using braces to build groups
		\codex{Now with grouping}{commands-without-restriction-2}
	\end{block}
	\inote{Emphasize on the missing space after {\LaTeX}, the expanded TT-Font
	and the full centering of all the text.}
\end{frame}
\begin{frame}
	\begin{block}{{\LaTeX} environments}
		Used to modify blocks of content (e.g. centering text):
		\codex{Aligning text}{text-align}
	\end{block}
\end{frame}
\begin{frame}
	\begin{block}{{\LaTeX} packages}
		Additional plugins/modules called \emph{packages} can be included for
		more features like
		\begin{multicols}{2}
			\begin{itemize}
				\item inlcuding images
				\item extra-long tables
				\item mathematical formulas
				\item drawing chemical structure formulas
				\item spliting list in multiple rows (as demonstrated here)
			\end{itemize}
		\end{multicols}
	\end{block}

	\begin{block}{Using packages}
		Packages need to be included with
		\code{{\textbackslash}usepackage[optional arguments]\{package name\}}
		e.g. for easy including images:
		\code{{\textbackslash}usepackage\{graphicx\}}
	\end{block}

	A full list of available packages is available at
	\url{http://www.ctan.org/pkg}
\end{frame}

\subsection{More {\LaTeX} characteristics}
\begin{frame}
	\begin{block}{Whitespaces}
		Multiple whitespaces (e.g. space, tab-space, enter) are ignored in
		\LaTeX{}
	\end{block}
	\codex{Newlines can be added manually}{newlines}
\end{frame}

\begin{frame}
	\begin{block}{Comments}
		Additional text, that is not displayed at the end can be added
		after \%
	\end{block}
	\codex{Comment example}{comment}
\end{frame}

\begin{frame}
	\begin{block}{Special characters}
		{\LaTeX} uses special characters e.g. ''\textbackslash'' to start
		commands or ''\%''
		for comments. If these symbols should be printed, they need to be 
		escaped.
	\end{block}
	\codex{Special characters require escaping}{special-characters}

	\LaTeX{} provides support for $> 5000$ symbols which are listed at 
	\url{http://www.ctan.org/pkg/comprehensive}
\end{frame}

\subsection{Structuring text}
\begin{frame}
	\codex{Structures}{structures}
\end{frame}
\begin{frame}
	\codex{Table of contents}{tableofcontents}
	Further available commands are \lcs{listoffigures} and \lcs{listoftables}.
\end{frame}

\subsection{Lists}
\begin{frame}
	\codex{Enumeration}{enumerate}
\end{frame}
\begin{frame}
	\codex{Item list}{itemize}
\end{frame}
\begin{frame}
	\codex{Description}{description}
\end{frame}

\subsection{Font formatting}
\begin{frame}
	\begin{alertblock}{Warning}
		It is strongly discouraged to perform direct formatting of text! Later
		we will learn better methods. Nevertheless, these commands are needed
		and therefore listed here.
	\end{alertblock}

	The creation of new commands to wrap semantics into layouts is described
	later.
\end{frame}
\begin{frame}
	\codex{Different font sizes}{font-size}
\end{frame}

\begin{frame}
	\codex{Direct format of text}{font-format}
\end{frame}

\begin{frame}
	Fonts have to be prepared for {\LaTeX} therefor it can not use the normal
	fonts on the system.

	\begin{block}{Font types}
		{\LaTeX} provides a variety of fonts as listed on
		\url{http://www.tug.dk/FontCatalogue}
	\end{block}

	\begin{block}{True-Type fonts}
		System fonts (mostly TTF) can be incorporated in {\LaTeX} which is not
		explained here. Please refer to
		\href{http://en.wikibooks.org/wiki/LaTeX/Fonts}{the {\LaTeX} WikiBook}
	\end{block}
\end{frame}
