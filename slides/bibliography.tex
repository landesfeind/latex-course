\section{Bibliography with Bib\TeX{}}

\subsection{{\LaTeX} build in bibliography}
\begin{frame}
	Built in support for references using the \len{thebibliography} and
	\lcs{cite} command
	\codex{References using the build in bibliography}{thebibliography}
\end{frame}
\begin{frame}
	\codex{References using the build in bibliography (define cite name)}{thebibliography-names}
\end{frame}

\subsection{Using a Bib{\TeX} library}
\begin{frame}
	The more {\LaTeX} way is a separate library using {Bib\TeX}
	\codexBibtex{{Bib\TeX} library file}{bibtex-library}
	Commands are located where the final bibliography will be included.
\end{frame}
\begin{frame}
	\codex{Using the library in {\LaTeX}}{bibtex-bibliography}
\end{frame}
\begin{frame}
	Compiling {\LaTeX} with {Bib\TeX} requires extra effort
	\begin{columns}[onlytextwidth]
		\column{0.35\textwidth}
		\lstinputlisting[language=bash,keywords={pdflatex,bibtex}]{compile-with-bibtex.sh}
		\column{0.64\textwidth}
			\begin{enumerate}
				\item create basic output file\\
					\texttt{\tiny LaTeX Warning: Citation `lamport94' on ...
					undefined}
				\item include bibliography
				\item compile with the bibliography\\
					\texttt{\tiny LaTeX Warning: Label(s) may have changed. Rerun
					...}
				\item ensure the cross-references to be right
			\end{enumerate}
	\end{columns}
\end{frame}


\subsection{Using the \textbackslash{}cite command}
\begin{frame}
	\codex{Citing with optional argument}{cite-optional}
\end{frame}
\begin{frame}
	Default {\LaTeX} bibliography uses numerical references (often used in
	mathematics/computer science).

	\codex{\lp{natbib} package supports named citing}{cite-natbib}

	More possible commands are: \lcs{citep}, \lcs{citet*}, \lcs{citep*}

	\warn{Natbib does not work with numerically referenced bibliographies}
\end{frame}


\subsection{Selecting an appropriate {Bib\TeX} style}
\begin{frame}
	Different styles can be used to produce different views
	\begin{center}
		\begin{tabular}{l|l|l|l}
		\multicolumn{1}{c|}{Style Name} & \multicolumn{1}{c|}{Author Name Format} & \multicolumn{1}{c|}{Reference Format} & \multicolumn{1}{c}{Sorting} \\ \hline
		plain & Homer Jay Simpson & \#ID\# & by author \\ \hline
		unsrt & Homer Jay Simpson & \#ID\# & as referenced \\ \hline
		abbrv & H. J. Simpson & \#ID\# & by author \\ \hline
		alpha & Homer Jay Simpson & Sim95 & by author \\ \hline
		abstract & Homer Jay Simpson & Simpson-1995a &  \\ \hline
		acn & Simpson, H. J. & \#ID\# &  \\ \hline
		authordate1 & Simpson, Homer Jay & Simpson, 1995 &  \\ \hline
		apa & Simpson, H. J. (1995) & Simpson1995 &  \\ \hline
		named & Homer Jay Simpson & Simpson 1995 &  \\ \hline
		\end{tabular}
	\end{center}

	More styles at
	\url{http://www.cs.stir.ac.uk/~kjt/software/latex/showbst.html}

\end{frame}

\subsection{Creating your own style for {Bib\TeX}}
\begin{frame}
	\begin{block}{Bib{\TeX} style files}
		{Bib\TeX} libraries do not contain style information (only structured
		information). 

		Styles are stored as {\color{code}.bst} files (pure text).
	\end{block}

	\begin{block}{Creating an own style}
		Simply execute \code{latex makebst} on the command line and respond to
		the questions asked.

		\textbf{Note:} Answering the last question ''Shall I now run this batch
		job?'' with yes or executing it by yourself:
		\code{latex my-style.dbj}
		will result in a {\color{red}my-style.bst} (the actual style file).
	\end{block}

\end{frame}

\subsection{Editing bibliography graphically}
\begin{frame}
	JabRef may be used to edit a {Bib\TeX} library
	\begin{center}
		\mimage{ss-jabref.png}

		\url{http://jabref.sourceforge.net}
	\end{center}	
\end{frame}


\begin{frame}
	Also \href{http://www.zotero.org}{Zotero} can export {Bib\TeX}
	\begin{center}
		\mimage{ss-zotero-export.png}
	\end{center}
\end{frame}

\subsection{Bib{\TeX} is supported by many online platforms}
\begin{frame}
	As {\LaTeX} is used extensively in academia, many online sources support Bib{\TeX}.
	
	JabRef can import several outside sources and Google Scholar provides Bib{\TeX} code:
		\begin{center}
		\mimage{bibtex_google_scholar.pdf}
	\end{center}
\end{frame}
