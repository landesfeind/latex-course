\section{Create and install new classes and packages}

\subsection{The {\LaTeX} file ecosystem}
\begin{frame}
	\begin{block}{{\LaTeX} package files \code[inline]{*.sty}}
		packages may contain all commands from your preamble
	\end{block}
	\begin{block}{{\LaTeX} class files \code[inline]{*.cls}}
		Classes are included/selection via the \lcs{documentclass}
	\end{block}
\end{frame}

\subsection{Creating your own package}
\begin{frame}
	\begin{codeblock}{The beginning of the \code[inline]{latex-course.sty} file
			\hfill {\tiny (\href{run:latex-course.sty}{Code})}}
		\lstinputlisting[lastline=9]{latex-course.sty}
	\end{codeblock}
	\warn{The name of the file (without extension) and the name in
	 \lcs{ProvidesPackage} have to match}
\end{frame}

\subsection{Creating your own class}
\begin{frame}
	\begin{codeblock}{Starting a new class}
		\lstinputlisting[lastline=3]{gaug-thesis-proposal.cls}
	\end{codeblock}
	\warn{The name of the file (without extension) and the name in
	 \lcs{ProvidesClass} have to match}
\end{frame}

\begin{frame}
	\begin{codeblock}{Create your own variable}
		\lstinputlisting[firstline=5,lastline=6]{gaug-thesis-proposal.cls}
	\end{codeblock}
	Define your own variable and load it with a default value.
\end{frame}

\begin{frame}
	Document class options can be declared as own options or given to a base class.
	\begin{codeblock}{Declaring option}
		\lstinputlisting[firstline=8,lastline=12]{gaug-thesis-proposal.cls}
	\end{codeblock}
\end{frame}

\begin{frame}
	Document class options can be declared as own options or given to the
	loaded base class
	\begin{codeblock}{Declaring option}
		\lstinputlisting[firstline=8,lastline=15]{gaug-thesis-proposal.cls}
	\end{codeblock}
	Load a base class for your class.
\end{frame}

\subsection{Differences between class and package}
\begin{frame}
	\begin{block}{When is a package a class?}
		Rule of thumb: Are the commands to be used in other documents/classes?
		You have a package!
	\end{block}
\end{frame}

\subsection{The special @-letter}
\begin{frame}
	\begin{block}{In normal text}
		The @ is generally assumed to be an ''other'' character $\rightarrow$
		can not be used in command names.
	\end{block}

	\begin{block}{In classes and packages}
		The @ is a normal letter and \textbf{can} be used in command names.
		$\rightarrow$ used to protect class internal methods 
	\end{block}

	Good practice: Use @-commands to store data
\end{frame}

\subsection{installing a {\LaTeX} package manually}
\label{installing-a-latex-package}
\begin{frame}
	\begin{itemize}
		\item Many packages are shipped with the major {\LaTeX} distributions
		\item if not a \code[inline]{*.sty} package file might be available for
			download
		\item a few require installation (e.g. \lp{subfiles})
	\end{itemize}
	
	\begin{columns}[c]
		\column{0.5\textwidth}
		\begin{block}{Required files for installation (e.g. \lp{subfiles})}
			\begin{itemize}
				\item subfiles.dtx \item subfiles.ins
			\end{itemize}
		\end{block}
		\column{0.5\textwidth}
		\begin{block}{Compiling the package}
			\code{latex subfiles.ins}
			will produce the file \code[inline]{subfiles.sty}
		\end{block}
	\end{columns}
\end{frame}
