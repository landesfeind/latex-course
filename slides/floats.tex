\section{Floating environments}
\subsection{On floats}
\begin{frame}
	Floats are containers for images, tables, and user-defined contents
	\begin{columns}[c]
	\column{0.4\textwidth}
	\begin{itemize}
		\item auto-placed: floats do not appear at the position they are
			defined
		\item un-broken: a float can/will not stretch over more than 1
			page
	\end{itemize}
	\column{0.59\textwidth}
	Available by default: 
	\begin{itemize}
		\item \len{figure}
		\item \len{table}
	\end{itemize}
	\end{columns}
\end{frame}

\begin{frame}
	\begin{block}{Caption}
		\lcs{caption} adds a caption to the current float, via
		\code{\textbackslash{}caption[short variant]\{My caption\}}
	\end{block}

	\begin{block}{Float counter}
		Every float environment has an own counter for later reference.
	\end{block}
\end{frame}

\subsection{The figures float}
\begin{frame}
	Used for graphics...
	\codex{Graphic in a figure float}{figure}
\end{frame}


\subsection{The table float}
\begin{frame}
	... And for tables.
	\codex{A table in a float}{table}
\end{frame}

\subsection{Placing the floats}
\begin{frame}
	\codex{Defining a specific placement}{floats-placement}

	\begin{description}
		\item[h] places the float approximately here
		\item[t] at the top of a page
		\item[b] at the bottom of a page
		\item[p] put floats together on a page containing only floats
		\item[H] put the float exactly here (requires \lp{float} package)
	\end{description}
\end{frame}
\subsection{More layout options with floats}
\begin{frame}
	\codex{Captions on the side}{floats-sidecap}
\end{frame}
\begin{frame}
	\codex{Wrapping text around figures}{floats-wrapfig}
\end{frame}
\begin{frame}
	\codex{Having subfloats}{floats-subfloats}
\end{frame}

\subsection{Creating an index for floats}
\begin{frame}
	Creating an index of figures and tables is trivial:
	\codex{Index of the figures used}{listoffigures}
	Lists of tables are created with the \lcs{listoftables}
\end{frame}
\begin{frame}
	Also own floats can be printed as list
	\codex{List of own floats}{float-listof}
\end{frame}
\begin{frame}[plain]
	\warn{List generally are not displayed in the the table of contents!}
	\codex{Including list into the table of
	contents}{tableofcontents-include-other-lists}
	The \lcs{phantomsection} is required for the correct page number.
	\lcs{addcontentsline} creates label to be references.
\end{frame}


\subsection{Creating new floats}
\begin{frame}
	\codex{Creating a custom float for computer source code}{floats-custom-np}
\end{frame}
\begin{frame}
	\begin{itemize}
		\item \lcs{floatstyle} - define the default style for new floats\\
			\begin{itemize}
				\item plain - normal layout
				\item plaintop - normal layout with caption above float
				\item boxed - a box around the float with caption outside
				\item ruled - using rules (see example)
			\end{itemize}
		\item \lcs{newfloat} - creates the new float with the given name\\
			\code{\textbackslash{}newfloat\{name\}\{placement
					identifiers\}\{listname\}[outer
				counter name]}
				Captions are written to the \texttt{listname} file to be used in
				lists. If outer counter is given it will prefix the counter
				(e.g. chapter, section)
		\item \lcs{floatname} - set name for new float
	\end{itemize}
\end{frame}

\subsection{Layout captions}
\begin{frame}
	Extended format specifications via the
\href{http://www.ctan.org/pkg/caption}{\lp{caption}}

	\codex{Decrease caption indent}{captionsetup}
\end{frame}

