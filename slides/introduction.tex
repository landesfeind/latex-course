\section{Introduction}
\subsection{About \LaTeX}

\begin{frame}
	\begin{columns}[onlytextwidth]
		\column{0.59\linewidth}
			\mquote{Wikipedia}{\LaTeX{} is a document markup language and document preparation system for
			the \TeX{} typesetting program.}
			\begin{itemize}
				 \item 1987 - Donald E. Knuth developed \TeX{} (Setter) for writing his book
				''The art of computer programming''
				\item  1994 - \LaTeX{} (Layout) is developed by Leslie Lamport for
					easier use of {\TeX}
				\item  Active community develops \LaTeX{} makros and additional packages.
			\end{itemize}
		\column{0.39\linewidth}
			\mimage{latex_stack.pdf}
	\end{columns}
\end{frame}

\subsection{Why developing texts in \LaTeX{}}
\begin{frame}
	\begin{block}{Devide writing and layout}
		While writing \LaTeX{} the final layout is not visible immediately
		(opposed to WYSIWYG word processors). 
		\begin{itemize}
			\item focuses the writer on the writing, thus more productive
				\inote{writer is not distracted by the layout}
			\item special typesetting (e.g. sigils)
				\inote{many letter \todo{Folgen} are better readable if the 
					were rendered special e.g. ff}
		\end{itemize}
	\end{block}

	\begin{block}{Portability}
		Simple text-files can be opened with any editor in any operating 
		system.
		\inote{ever tried to open a docX file from work on your home PC with
		an older MS Office version?}
		\inote{text files can also be opened in 10 years}
	\end{block}
\end{frame}

\subsection{Resources}
\begin{frame}

	\begin{itemize}
		\item CTAN \url{http://www.ctan.org/}\\
			the \emph{C}omprehensive \emph{T}eX \emph{A}rchive \emph{N}etwork
		\item LaTeX Wikibook\\
			\url{http://en.wikibooks.org/wiki/LaTeX}

		\item \LaTeX{} templates\\
			\url{http://www.latextemplates.com}
	\end{itemize}
\end{frame}


\subsection{Writing a \LaTeX{} document}
\begin{frame}
\begin{center}
	\mimage[width=0.6\textwidth]{latex_compile_process.pdf}
\end{center}
\end{frame}


\subsection{Available Software}
\begin{frame}
	\frametitle{\LaTeX packages}

	\begin{block}{Texlive (Linux)}
		Available in all distributions, e.g. Ubuntu, Suse Linux. Install it
		from your software center (attention: several packages may be
		required).
	\end{block}
	\begin{block}{MiKTeX (Windows)}
		Freely available at \url{http://www.miktex.org}.
	\end{block}	
	\begin{block}{MacTeX (Mac OS)}
		Freely available at \url{http://www.tug.org/mactex}.
	\end{block}
\end{frame}

\subsection{\LaTeX{} editors}
\begin{frame}

	\inote{Remember: What is the difference between a text-editor and -processor?}
	Every simple text-editor can be used but more comfortable:

	\begin{columns}[T,onlytextwidth]
	\column{0.49\textwidth}
		\begin{block}{TeXMaker}
			\url{http://www.xm1math.net/texmaker}
			\begin{itemize}
				\item free and cross-platform editor
				\item syntax highlightning
				\item compile via mouse click
				\item displays previews
			\end{itemize}
		\end{block}
	\column{0.49\textwidth}
		\begin{block}{Kile}
			\url{http://kile.sourceforge.net}
			\begin{itemize}
				\item integrated {\LaTeX} environment
				\item easily installed on all Linux distributions
			\end{itemize}
		\end{block}
	\end{columns}

	\vfill

	\begin{block}{Mobile devices}
		Editors for mobile devices can be found in the Play/App Store
		(may require web-access to compile the documents).
	\end{block}
\end{frame}
\subsection{Creation using the Texmaker IDE}
\begin{frame}
	\begin{enumerate}
		\item Start Texmaker
		\item create a new document
		\item insert basic information
		\item compile and view
	\end{enumerate}

	\mimage{texmaker-intro.pdf}
\end{frame}

\subsection{A first example}
\begin{frame}
	\codex{Basic document}{basic-document-np}
	\begin{itemize}
		\item \lcs{documentclass} defines the type of the text e.g. book,
			article, report
		\item \len{document} contains the actual text (the part
			before is called \emph{preamble})
		\item \lcs{maketitle} produces a basic heading
	\end{itemize}
\end{frame}
\subsection{Names and ...}
\begin{frame}
	\begin{description}
		\item[package] is an additional \LaTeX{} plugin with a special purpose
			(e.g. for images, specific mathematical/chemical formulas)
		\item[command] specific modification of the document by the means of
			the given command, e.g. emphasizing a text or bold/italic font.
		\item[environment] specific modification of a longer part of the text,
			e.g. centering a paragraph.
	\end{description}
\end{frame}



\subsection{Compiling documents}
\begin{frame}
	Resulting file formats are
	\href{http://en.wikipedia.org/wiki/Page_description_language}{\emph{page description
	format}} files:
	\begin{block}{File formats}
		\begin{description}
			\item[DVI] Device independent file format\\
				\url{http://de.wikipedia.org/wiki/Device_independent_file_format}
			\item[PS] PostScript\\
				\url{http://de.wikipedia.org/wiki/PostScript}
			\item[PDF] Portable Document Format\\
				\url{http://en.wikipedia.org/wiki/PDF}
		\end{description}
	\end{block}

	\begin{block}{Using PDFLaTeX on the command line}
		\code{pdflatex document.tex}
		It is preferable to execute the command twice because specific parts
		of the document are first correct after the second run.
	\end{block}
\end{frame}
