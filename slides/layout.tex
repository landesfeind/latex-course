\section{Defining an own layout}

\subsection{Specify the page layout}
\begin{frame}
	\begin{block}{Documentclass options}
		The different document classes provide options for basic setup
	\end{block}
	\begin{itemize}
		\item oneside, twoside - used in books and reports to change the area to be
			printed (also page headers can change)
		\item onecolumn, twocolumn - change to twocolumn layout
		\item 10pt, 11pt, 12pt - specify the default font size
		\item a4paper, letterpaper - for ouput size
	\end{itemize}
\end{frame}

\subsection{Changing text layout}
\begin{frame}
	\begin{block}{Setting the text font}
		{\LaTeX} provides many fonts\footnote{see {\LaTeX} Font Catalogue at \url{http://www.tug.dk/FontCatalogue/}}
	\end{block}
	\codex{Changing the font for the whole document}{arev-np}
\end{frame}	
\begin{frame}
	\codex{Changing font for parts}{rmdefault-np}
	\begin{mitemize}
		\item \lcs{rmdefault} (default roman font)
		\item \lcs{ttdefault} (default monospace font)
		\item \lcs{sfdefault} (default serif font)
	\end{mitemize}
\end{frame}
\begin{frame}
	\codex{Change paragraph indention (hard way)}{indention}
	\warn{Changing \lcs{parskip} \emph{may} have side-effects}
\end{frame}
\begin{frame}
	Using specialized documentclasses can help
	\codex{Paragraph indention with KOMA-Script}{indention-koma-np}
\end{frame}
\begin{frame}
	\codex{Setting linespacing}{linespacing}
\end{frame}
\begin{frame}
	\codex{Additional handling for lists}{linespacing-lists}
\end{frame}

%\section{Customizing headings and captions}
%\begin{frame}
%	Two packages can be used for an easy changing of headings (chapter,
%	section, \ldots) and captions.
%
%	\codex[no-preview]{Using the \lp{titlesec}, \lp{caption} packages}{titlesec-caption}
%\end{frame}

\subsection{Setting page margins}
\begin{frame}
	\begin{block}{The \lp{geometry} package}
		supports the specification of document-wide margins
	\end{block}
	\codex{Setting page margins (1)}{geometry-1-np}
\end{frame}
\begin{frame}
	\codex{Setting page margins (2)}{geometry-2-np}
\end{frame}

\subsection{Custom header and footer}
\begin{frame}
	\begin{block}{Build-in methods}
		{\LaTeX} provides pagestyles \texttt{empty}, \texttt{plain},
		\texttt{headings}, \texttt{myheadings}.
	\end{block}
	\codex{The \lcs{pagestyle} command}{pagestyle}
\end{frame}
\begin{frame}
	A more convenient solutions with the \lp{fancyhdr} package
	\codex{The \lp{fancyhdr} supports the layout of head and foot lines}{fancyhdr}
\end{frame}
\begin{frame}
	\begin{block}{Changing header and footer}
		\begin{mitemize}
			\item \lcs{fancyhead} \item \lcs{fancyfoot}
		\end{mitemize}
	\end{block}
	\begin{block}{Select part to change}
		\begin{mitemize}
			\item \ubf{Left} \item \ubf{Center} \item \ubf{Right} 
			\item \ubf{Left} on \ubf{odd} pages (LO)
			\item \ubf{Left} on \ubf{even} pages (LE)
			\item \ubf{Right} on \ubf{odd} pages (RO)
			\item \ubf{Right} on \ubf{even} pages (RE)
			\item \ubf{Center} on \ubf{odd} pages (CO)
			\item \ubf{Center} on \ubf{even} pages (CE)
		\end{mitemize}
		Can also be combined using comma
		\code{\lcs{fancyhead}[LE,RO]\{\lcs{leftmark}\}}
	\end{block}
	\begin{block}{Useful commands}
		\begin{mitemize}
				\item \lcs{leftmark} 
				\item \lcs{rightmark}
					current chapter name in different styles
				\item \lcs{thechapter} current chapter number
				\item \lcs{chapter} translation of ''chapter'' in the current
					language
				\item \lcs{thesection} current section number
			\item \lcs{thepage} current page number
		\end{mitemize}
	\end{block}
\end{frame}
\begin{frame}
	\begin{block}{Specify profiles}
		Define sets of rules that can later be activated with \lcs{pagestyle}
	\end{block}
	\codex{Define pagestyle ''default''}{fancyhdr-profile}
\end{frame}
\begin{frame}
	\begin{block}{Override profiles}
		\lcs{fancypagestyle} allows to override  existing pagestyles
	\end{block}
	\codex{Overriding ''plain''}{fancyhdr-profile-override}
\end{frame}
\begin{frame}
	\begin{block}{Common problem: pagestyle on specific pages}
		Specific pages/commands change the pagestyle for current site to
		``plain'', e.g. table of contents and the \lcs{chapter} command.
	\end{block}

	\begin{block}{Possible scenarios}
		\begin{enumerate}
			\item table of contents \textbf{and} chapter should pages look
			like all other pages $\rightarrow$ override the ``plain'' style (see
			slide \pageref{codex:fancyhdr-profile-override}).
		\item TOC should be one style, but chapter pages another $\rightarrow$
			define ``plain'' to your needs; issue chapter to call
			\lcs{thispagestyle}
		\end{enumerate}
	\end{block}
\end{frame}
\begin{frame}
	\codex{Plain TOC and fancy chapter page}{fancyhdr-chapter-np}
	\warn{Do not try to re-new the \lcs{chapter} because TOC and others use it
	to define their layout etc.}
\end{frame}
\begin{frame}
	\begin{block}{Robust solution}
		Copy the baseclass (e.g. \texttt{report.cls}) and modify the needed
		parts
	\end{block}
	\begin{codeblock}{Excerpt from report.cls}
		\lstinputlisting{report-part-style-chapter-fail.tex}
	\end{codeblock}
	\warn{If you are working with the base classes a lot of {\TeX} is waiting
		to corrupt your soul!!}
\end{frame}

\begin{frame}
	\frametitle{Header and footer in KOMA-Script documents}

	KOMA script provides its own header and footer commands that are more
	robust than \lp{fancyhdr}

	\codex{Usage of \lp{scrlayer-scrpage}}{srclayer-scrpage}

	See also the KOMA-Script documentation starting at section 5.4\\
	\url{http://mirrors.ctan.org/macros/latex/contrib/koma-script/doc/scrguide.pdf}

\end{frame}



\begin{frame}
	\begin{block}{Determining the last page}
		{\LaTeX} provides no command to determine the last page
	\end{block}
	\codex{Cheating with labels to get the last page}{lastpage-np}
\end{frame}

\subsection{Layout pagenumbers}
\begin{frame}
	\codex{Refresh the page counter}{pagecounter}
	\begin{mitemize}
		\item arabic - normal digits
		\item roman - lowercase roman digits
		\item Roman - uppercase roman digits
		\item alph - lowercase letters
		\item Alph - uppercase letters
	\end{mitemize}
\end{frame}
	
\subsection{Last warning}
\begin{frame}
	\begin{block}{Be sparse}
		\LaTeX{} already provides a very good (because common) typesetting and
		implements many useful rules. {\color{red}Don't mess with it!}
	\end{block}

	\begin{block}{Inform yourself}
		There are several good books on typesetting, e.g. 
		\begin{itemize}
			\item Basic Book Design by Kehoe (also a Wikibook)\\
				\url{http://en.wikibooks.org/wiki/Basic_Book_Design}
			\item KOMA-Script documentation (pages 23 to 46)\\
				\url{http://mirror.ctan.org/macros/latex/contrib/koma-script/doc/scrguien.pdf}
		\end{itemize}
	\end{block}

For more commands read the ''{\LaTeX} layout manual'' at
\url{http://www.texdoc.net/texmf-dist/doc/latex/layouts/layman.pdf}

\end{frame}
