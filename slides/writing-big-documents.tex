\section{Writing large documents}

\subsection{Using the \textbackslash{}input command}
\begin{frame}
	\LaTeX{} supports the creation of large documents (e.g. thesis, books) by
	splitting the document into smaller ones.

	\codex{Including documents}{input-simple-np}

	Alternatively use \lcs{include} which is short for
	\code{\textbackslash{}newpage\textbackslash{}input\{your-document.tex\}}
\end{frame}

\begin{frame}
	\begin{block}{Drawbacks using \lcs{input} or \lcs{include}}
		\begin{itemize}
			\item When a document grows, the time to compile increases (e.g. this
				presentation $\approx$6.5 minutes).
			\item Compiling an included file results in failure because the preamble is
				missing.
		\end{itemize}
	\end{block}
	\begin{block}{Required solution/package}
		\begin{enumerate}
			\item split the document into several files
			\item single files compilable
			\item include the preamble of main document
		\end{enumerate}
	\end{block}
\end{frame}

\subsection{The subfiles package}
\begin{frame}
	\lp{subfiles} supports exactly this behaviour
	\codex{Main document using \lp{subfiles} package}{subfiles-np}
	{\tiny \warn{The \lp{subfiles}
			(\href{http://www.ctan.org/pkg/subfiles}{CTAN-link}) package is
			maybe not included in your {\LaTeX}. See slide
\pageref{installing-a-latex-package} for instructions.}}
\end{frame}
\begin{frame}
	\codex{Subfile containing the introduction}{subfiles-introduction-np}
\end{frame}

\subsection{Usefull packages}
\begin{frame}
	The \lp{todonotes}\footnote{\url{http://www.ctan.de/pkg/todonotes}} helps
	during development
	\codex{Adding a todo-note}{todonote}
\end{frame}

\begin{frame}
	Add explaining
	margin-notes\footnote{\url{http://www.ctan.org/pkg/marginnote}} to your document

	\codex{Margin-notes with the marginnote package}{marginnote}
\end{frame}

\begin{frame}
	Adding appendices with special numbering

	\codex{Appendix for very long tables}{appendices}
\end{frame}


\subsection{Use folders}
\begin{frame}
	\begin{itemize}
		\item folder for your project\\
			contains only *.tex files
		\item ''images''--folder\\
			contains all graphics
		\item \ldots
		\item ''tmp''--folder\\
			contains the output files
	\end{itemize}
\end{frame}

\subsection{Command line {\LaTeX}}
\begin{frame}
	\code{pdflatex document.tex}

	\begin{block}{Specific output directory}
		\code{pdflatex -output-directory tmp document.tex}
		All temporary and output files are put there and a folder can be
		deleted easily.
	\end{block}

	\begin{block}{Environment variables}
		\code{TEXINPUTS} contains the directories where pdflatex search input
		files.
		\code{TEXINPUTS=".:lib:~/LaTeX:\$TEXINPUTS"}
	\end{block}
\end{frame}

